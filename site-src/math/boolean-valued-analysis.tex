---
title: Boole値解析入門
date: 2016/07/11 11:00:00 JST
author: 石井大海
description: |
  集合論の分野で,**強制法**はモデルを構成する主要な方法の一つであり,種々の命題の**無矛盾性・独立性の証明**に用いられています.
  一方,Kunen\cite{Kunen:1980}でも少しだけ触れられているように,強制法は無矛盾性証明だけでなく,**$\ZFC$の定理を証明する手法**としても用いることが出来ます.
  本稿では,この側面の用いられ方の一つである**Boole値解析**の手法を採り上げます.
  これは強制法の同値な定式化である**Boole値モデル**の手法を使って,初等解析の結果を測度論や関数解析の定理に読み替える(!)という物で,例えば「確率変数」が「あるモデルの中の実数」と一対一に対応したりします.
  Boole値モデルを定式化した一人であるScottらが初期から提案していたものですが,今回はTakeuti\cite{Takeuti:2012cv}の説明に基づきます.

latexmk: -lualatex
tag: Boole値モデル,集合論,解析学,測度論,Boole値解析
katex: true
draft: true
macro:
  M: '\mathbb{M}'
  Add: '\mathop{\mathsf{Add}}'
  img: '\mathbin{\text{``}}'
---
\documentclass[a4j,leqno]{ltjsarticle}
\usepackage[hiragino-pron]{luatexja-preset}
\usepackage{mystyle}
\usepackage{luatexja-otf}
\usepackage{dsfont}
\DeclareMathAlphabet{\mathrsfs}{U}{rsfso}{m}{n}
\renewcommand{\mathscr}[1]{\mathup{\mathrsfs{#1}}}
\usepackage{fixme}
\usepackage[super]{nth}
\usepackage[bookmarksnumbered,pdfproducer={LuaLaTeX},%
            luatex,psdextra,pdfusetitle,pdfencoding=auto]{hyperref}
\usepackage[backend=biber, style=math-numeric]{biblatex}
\addbibresource{myreference.bib}
\renewcommand{\emph}[1]{\textsf{\textgt{#1}}}

\title{Boole値解析入門}
\author{石井大海}
\date{2016/07/11 11:00:00 JST}
\begin{document}
\maketitle

\begin{abstract}
 集合論の分野で,\emph{強制法}はモデルを構成する主要な方法の一つであり,種々の命題の無矛盾性・独立性の証明に用いられています.
 一方,Kunen~\cite{Kunen:2011}でも少しだけ触れられているように,強制法は無矛盾性証明だけでなく,\emph{$\ZFC$の定理を証明する手法}としても用いることが出来ます.
 本稿では,この側面の用いられ方の一つである\emph{Boole値解析}の手法を採り上げます.
 これは強制法の同値な定式化である\emph{Boole値モデル}の手法を使って,初等解析の結果を測度論や関数解析の定理に読み替える(!)という物で,例えば「確率変数」が「あるモデルの中の実数」と一対一に対応したりします.
 Boole値モデルを定式化した一人であるScottらが初期から提案していたものですが,今回はTakeuti~\cite{Takeuti:2012cv}の説明に基づきます.
\end{abstract}

\section{定理証明技法としての強制法}
集合論において,\emph{強制法}はモデルを構成する主要な手法の一つだが,専ら無矛盾性証明のための道具として用いられることが多い.
強制法の基礎理論については,以前『Boole値モデルと強制法』という記事~\cite{Ishii:2016mj}で紹介したが,今回は\emph{定理証明技法としての強制法}について採り上げる.
そのため,詳細には立ち入らないが,まずはウォームアップと題して違う手法を採り上げて,それを通して強制法の基本的な事項を採り上げていきたい.

\subsection{ウォームアップ:絶対性と強制法のあわせ技による定理の証明}
本題のBoole値解析の話に入る前に,Kunen~\cite{Kunen:2011}に載っている強制法による定理証明の方法を見てみよう.

\begin{definition}
 \begin{enumerate}
  \item 順序集合$(T, <)$が\emph{木}であるとは任意の$x \in T$に対して${\downarrow}x \defeq \Set{ y \in T | y < x}$が$<$に関し整列集合となること.
  \item $\rank_T(x) \defeq \otp({\downarrow}x)$, $T_\alpha \defeq \Set{ x \in X | \rank_T(x) = \alpha }$, $\height(T) \defeq \sup_{x \in X} (\rank_T(x)+1)$.
  \item 木$(T, <)$が\emph{Suslin} $\defs$ $T$は高さ$\omega_1$の木で,$|T_\alpha| \leq \aleph_0$かつ$T$は順序型$\omega_1$の鎖を持たない.
  \item Suslin木$T$が\emph{well-pruned} $\defs$任意の$x \in T$と$\alpha < \omega_1$に対し,$T_\alpha \cap {\uparrow}x \neq \emptyset$.
 \end{enumerate}
\end{definition}

\begin{theorem}
 $T$がSuslin木の時,狭義単調写像$\varphi: T \rightarrow \R$は存在しない.
\end{theorem}

この定理に使うのは次の補題だ:

\begin{lemma}\label{lem:no-omega-1-incr-reals}
 実数の長さ$\omega_1$の狭義増大列は存在しない.
\end{lemma}
\begin{proof}
 $\Braket{x_\alpha | \alpha < \omega_1}$が実数の狭義増大列だとする.
 このとき,$\R$における$\Q$の稠密性から,各$\alpha < \omega_1$に対し$x_\alpha < r_\alpha < x_{\alpha+1}$となるような$\Braket{ r_\alpha \in \Q | \alpha < \omega_1}$が取れる.
 すると,取り方からこの$r_\alpha$たちは互いに相異なるが,これは$\Q = \aleph_0$に矛盾. \qed
\end{proof}

えっ,そもそも「順序型$\omega_1$の鎖を持たない」がSuslin木の定義に入ってるのに,どう使うの?と思うかもしれない.
実は,$V$におけるSuslin木$T$に対して,より大きな宇宙$V[G]$で$T$がSuslin木でなくなるような宇宙がある.

\begin{fact}
 任意のSuslin木$T$に対し,その部分木$T^*$でwell-prunedなものが取れる.
\end{fact}

これを別として,

\begin{proof}
 そのような$\varphi \in V$が取れたと仮定し矛盾を導く(\emph{背理法}).
 $G$を$(T, >)$-生成フィルターとする.
 $T$の各レベルが高々可算であることから,$T$は強制法としてc.c.c.を持ち,特に$V$と$V[G]$で$\omega_1$の値は変わらないことに注意する.
 特に,$T$は$V[G]$でも$\omega_1$-木である.

 いま,$T$がwell-prunedである事から,各$\alpha < \omega_1$に対し$T_\alpha$は稠密である.
 すると,$G \cap T_\alpha \neq \emptyset$かつ$G$が上に閉(つまり$T$の順序についていえば下に閉)であることから,$G$は$T$を通る道であり,$T$が$V[G]$でも$\omega_1$-木であることから$\otp(G) = \omega_1$である.
 狭義単調性は推移的モデルの間で絶対的で,特に$\varphi$は$V[G]$でも狭義単調写像であることに気を付ければ,$\varphi \mathbin{\text{``}} G \subseteq \R$は実数の順序型$\omega_1$の真の増大列である.
 これは補題~\ref{lem:no-omega-1-incr-reals}に反する. \qed
\end{proof}

\printbibliography[title=参考文献]
\end{document}
