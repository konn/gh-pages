---
title: 可算推移的モデルの存在について
author: 石井 大海
tag: 数学,数理論理学,集合論,強制法
description: ZFの可算推移的モデルの存在がCon(ZF)よりも真に強いことに関する説明です.
date: 2015/12/03 00:57:26 JST
latexmk: -lualatex
---
\documentclass[a4j]{ltjsarticle}
\usepackage[hiragino-pron]{luatexja-preset}
\usepackage{amsmath,amssymb}
\usepackage{epstopdf}
\usepackage{mystyle}
\usepackage{graphicx}
\usepackage{atbegshi}
\AtBeginShipoutFirst{\special{pdf:tounicode 90ms-RKSJ-UCS2}}
\usepackage[pdfauthor={石井大海},colorlinks=true,%
            pdftitle={可算推移的モデルの存在について}]{hyperref}
\usepackage[backend=biber, style=numeric]{biblatex}
\addbibresource{myreference.bib}
\usepackage{bm}
\usepackage{epstopdf}
\usepackage{cases}

\title{可算推移的モデルの存在について}
\author{石井大海}


\begin{document}
\maketitle

集合論の様々なモデルを構成する方法として\textbf{強制法}があります.強制法の流儀には複数ありますが,その中の一つに$\ZFC$の\textbf{可算推移的モデル}(\textit{c.t.m.})を取る方法があり,Kunen\cite{Kunen:2011}でもこの方法が採用されています.

実はこの「$\ZFC$の可算推移的モデルの存在」は「$\ZF(\mathrm{C})$の無矛盾性(=$\ZF(\mathrm{C})$の集合モデルの存在)」よりも強い仮定です(つまり$\Con(\ZF) \not\rightarrow \exists \text{c.t.m.}$)\footnote{ですので,Kunenでは「$\Con(ZFC)$からの相対無矛盾性を示す際には,厳密には反映原理で議論を展開するのに十分な$\ZFC$の有限部分を取ってきてそのc.t.m.を取ることになる」という説明がされています.}.このことはKunen\cite{Kunen:2011}や新井\cite{Arai:2011}で言及されていますが,具体的に何故なのかは触れられていません.
以下では,この辺りの議論について少し詳しく書いてみます.

\section{典型的な誤解とそれが誤解である手短な説明}
「$\Con(\ZF)$を仮定しているのだから集合モデルが存在するので,L\"{o}wenheim-Skolemの定理で可算な初等部分構造を取ってMostwski崩壊で$(M, \in)$の形にすればよい」というのがよくある誤解です.

具体的にどの部分が誤解なのかというと,「Mostwski崩壊で」の所が間違っています.Mostwski崩壊の主張をよく思い出してみましょう:

\begin{theorem}[Mostwski]
 $(M, E)$を外延的かつ整礎的な構造とする.このとき$(M, E) \cong (S, \in)$となる推移的集合$S$が存在する.
\end{theorem}

$\Con(ZF)$から存在する可算モデルを$(M, E)$としましょう.$=$は論理記号だと思ってしまえば,外延性の公理から$(M, E)$が外延的であることは良いでしょう.基礎の公理(正則性の公理)が成り立つから$E$は$M$上整礎なので,条件が成立して……と進めたくなりますが,実はここが間違っています.基礎の公理が$(M, E)$で成り立つ,ということは次の論理式が成り立つということです:

\[
 \forall A \in M\, [\exists x \in M\,(x \mathrel{E} A) \rightarrow \exists x \mathrel{E} A \, \forall y \mathrel{E} A\, (y \not\mathrel{E} x)]
\]

対して,$(M, E)$が整礎構造である,というのは,
\[
 \forall A \subseteq M\, [\exists x \in M\,(x \in A) \rightarrow \exists x \in A \, \forall y \in A\, (y \not\mathrel{E} x)]
\]
ということでした.これを見比べてみれば,Mostwski崩壊定理が求めているのはいわば「$(M, E)$が$\in$-整礎である」という条件であるのに対し,「$(M, E)$が基礎の公理を満たす」というのは「$(M, E)$が$E$-整礎である」ことを主張していることになります\footnote{また,厳密には$A$の範囲が$\subseteq$か$\in$かという差もあります.まあ,基礎の公理と,$M$が$E$の意味で空でない部分集合が$E$-極小元を持つことの同値性は初歩的な議論で出来るので良いと思います}.したがって,完全性定理とL\"{o}wenheim-Skolemの定理により得られた可算モデルにMostwski崩壊定理を適用することは出来ない訳です.

\section{真に強いことの証明}
以上の議論により,Mostwski崩壊による常套手段を$\ZFC$全体のモデルに適用してc.t.m.を得ることは出来ないということがわかりました.

しかし,それでも他の方法で取れる可能性はあるのではないか?と云う疑問が湧いてきます.そこで,以下では,c.t.m.の存在が$\Con(ZF)$よりも強いことを示します.
以下の議論については,くるるさん(\href{https://twitter.com/kururu_goedel}{@kururu\_goedel})から本質的な示唆\footnote{\url{https://twitter.com/kururu_goedel/status/514174894779953152}}を頂きました.ありがとうございます.

そこで,$\Con(\ZFC) \rightarrow \exists \text{c.t.m.}$を仮定して矛盾を導きましょう.そもそも$\ZFC + \Con(ZFC)$が矛盾する場合はつぶれてしまって考える意味がないので,$\Con(\ZFC + \Con(ZFC))$としましょう.するとG\"{o}delの第二不完全性定理および完全性定理から,$M \models \ZFC + \Con(ZFC) + \neg \Con(\ZFC + \Con(ZFC))$を満たすモデル$M$が存在します.今,$\Con(\ZFC) \rightarrow \exists \text{c.t.m.}$を仮定しているので,$M$の中で$\ZFC$の可算推移的モデル$N \subseteq M$が取れます.ここで,「可算」「推移的」「$\subseteq$」はいずれも$M$をユニヴァースと見た時のものであることに注意しましょう.とはいえ,これ以後$M$の外に出ることはないので,$M$をユニヴァースだと思ってしまって,以下$\in$は$M$における$\epsilon$-関係であるとして議論を進めることにします.

すると,$N \models \ZFC$かつ$\neg \Con(ZFC + \Con(ZFC))$より$N \models \neg \Con(ZFC)$となります.すると,$\ZFC$の有限個の公理$\varphi_1, \dots, \varphi_n$があってそこから矛盾が出ます:

\[
N \models \neg \Con(\varphi_1 \wedge \dots \wedge \varphi_n)
\]

この時,「論理式」「有限」「$\ZFC$の公理」「矛盾」の概念はそれぞれ推移的モデルについて絶対なので\footnote{絶対性の議論については拙稿「\href{http://konn-san.com/math/absoluteness-cheatsheet.html}{絶対性チートシート}」を参照の事.論理式・証明図は$\mathrm{HF}$の元として実現出来,「証明である」「証明可能である」などの概念が算術的であることに注意すれば大丈夫です.},外側の$M$でも同じことが成立します:

\[
M \models \neg \Con(\varphi_1 \wedge \dots \wedge \varphi_n)
\]

しかし,他方で$M \models \Con(\ZFC)$でしたから,当然$M \models \Con(\varphi_1 \wedge \dots \wedge \varphi_n)$でなくてはなりません.これは矛盾です.

\nocite{Eda:2010}
\printbibliography[title=参考文献]
\end{document}
