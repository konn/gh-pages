---
title: 集合論の地質学4:下方有向性原理の証明
author: 石井大海
description: |
  集合論の地質学に関する記事の第四回目.
  今回はいよいよ,マントルや生成多宇宙の構造の大部分を確定させる基本定理である**下方有向性定理**を証明します.  
  [前回はこちら](/math/geology-bukovsky-theorem.html).
  [初回はこちら](/math/geology-ground-definability.html).
latexmk: -pdflua
tag: Boole値モデル,無矛盾性証明,強制法,公理的集合論,集合論,生成多元宇宙,集合論の地質学,Bukovskýの定理,下方有向性定理
date: 2017/12/11 21:00:00 JST
---
\documentclass[a4j,leqno]{ltjsarticle}
\usepackage[hiragino-pron]{luatexja-preset}
\usepackage{mystyle}
\usepackage{luatexja-otf}
\usepackage{dsfont}
\DeclareMathAlphabet{\mathrsfs}{U}{rsfso}{m}{n}
\renewcommand{\mathscr}[1]{\mathup{\mathrsfs{#1}}}
\usepackage{fixme}
\usepackage[super]{nth}
\usepackage[bookmarksnumbered,pdfproducer={LuaLaTeX},%
            luatex,psdextra,pdfusetitle,pdfencoding=auto]{hyperref}
\usepackage[backend=biber, style=math-numeric]{biblatex}
\addbibresource{myreference.bib}
\renewcommand{\emph}[1]{\textgt{\textsf{#1}}}
\usetikzlibrary{shapes,shapes.geometric,positioning,math,fpu}

\title{集合論の地質学4:下方有向性原理の証明}
\newcommand{\mantle}{\mathbb{M}}
\newcommand{\cc}{c.c.\ }
\newcommand{\M}{\mantle}
\newcommand{\gM}{g\mathbb{M}}
\newcommand{\gmantle}{\gM}
\newcommand{\DDG}{\mathord{\mathrm{DDG}}}
\newcommand{\sDDG}{\mathord{\mathrm{sDDG}}}
\author{石井大海}

\usepackage{amssymb}	% required for `\mathbb' (yatex added)
\usepackage{amsmath}	% required for `\gather*' (yatex added)
\begin{document}
\maketitle

\begin{abstract}
 本稿は集合論の地質学に関する記事の第四回目です:
 \begin{enumerate}[label={\arabic*.}]
  \item \href{http://konn-san.com/math/geology-ground-definability.html}{概観と基礎モデルの定義可能性}
  \item \href{http://konn-san.com/math/geology-mantle-and-ddg.html}{マントルの構造と下方有向性原理}
  \item \href{http://konn-san.com/math/geology-bukovsky-theorem.html}{Bukovsk\'{y}の定理──強制拡大の特徴付け}
  \item 下方有向性原理の証明(今回)
 \end{enumerate} 
 \emph{集合論の地質学}は,与えられた集合論の宇宙$V$の内部モデルがいかなる生成拡大になっているかを考える集合論の分野です.
 今回は,薄葉~\cite{Usuba:2017fp}で証明された基本定理である\emph{下方有向性原理}の証明を与えます.
\end{abstract}
\section{下方有向性定理の証明}
いよいよこのシリーズのメインデュッシュである,次の定理の証明を与えます:
\begin{theorem}[Strong Downwad Directed Grounds Theorem, Usuba 2017~\cite{Usuba:2017fp}]
 任意の集合個の$V$の基礎モデル$\Set{W_r | r \in X}$に対し,共通の基礎モデル$W \subseteq \bigcap_r W_r$が存在する.
\end{theorem}
この$\sDDG$から,全ての基礎モデルの共通部分であるマントル$\M$が$\ZFC$を満たす内部モデルとなることや,強制法で不変な最大のクラスであること,また$V$の生成多宇宙における包含関係と生成拡大の関係は一致することなどは\href{http://konn-san.com/math/geology-mantle-and-ddg.html}{第二回}で既に見ました.

\href{http://konn-san.com/math/geology-bukovsky-theorem.html}{前回}はこの証明の中で重要な役割を担うBukovsk\'{y}による$\kappa$-\cc{}拡大の特徴付けを与えました:
\begin{theorem}[Bukovsk\'{y}~\cite{Bukovsky:1973ph}]\label{thm:bukovsky}
 $\ZFC^{\leq \delta}$の推移的モデルの組$(W, V)$が$\delta$-大域被覆性質を持つことと,$V$が$W$の$\delta$-\cc{}生成拡大であることは同値.
\end{theorem}
この系として,次の補題が得られていました:
\begin{lemma}[$\delta$-大域被覆性質を持つモデルの一意性]\label{lem:global-covering-unique}
 $\mu \defeq (2^{<\delta})^+$とおく.
 $W, W' \subseteq V$が$\ZFC^{\leq\mu}$の推移的モデルで$(W, V)$および$(W', V)$が共に$\delta$-大域被覆性質を持つとする.
 もし$\power{<\mu}{2} \cap W = \power{<\mu}{2} \cap W'$なら$W = W'$となる.
\end{lemma}
また,一般の強制拡大は$\kappa$-擬基礎モデルと呼ばれるものになることが\href{http://konn-san.com/math/geology-ground-definability.html\#lem:ground-is-pseudoground}{初回の議論でわかって}いました:
\begin{lemma}\label{lem:ground-is-pseudoground}
 $\mathbb{P}$を擬順序,$G$を$(V, \mathbb{P})$-生成フィルター,$|\mathbb{P}| \leq \kappa$とする.
 この時$V$は$V[G]$の擬基礎モデルとなる.
 特に$(\kappa^{++})^V = (\kappa^{++})^W$であり,$V \subseteq W$は$\kappa^+$-被覆性質および$\kappa^+$-近似性質を満たす.
\end{lemma}
また,生成拡大に挟まれた$\ZFC$のモデルも生成拡大になる,というお馴染の次の事実も使います:
\begin{fact}[中間拡大補題]\label{lem:interm-ext}
 $V \subseteq W \subseteq V[G]$が全て$\ZFC$の推移的モデルで$V[G]$が$V$の生成拡大なら,$W$は$V$の生成拡大で,$V[G]$も$W$の生成拡大となる.
\end{fact}
ところで,上の形の主張であれば,牛刀ですがBukovsk\'{y}の定理から従うので余力があれば示してみてください.

これらを踏まえて,基本的には求める$W \subseteq \bigcap_{r \in X} W_r$は十分大きな$\mu$に対して$\mu$-大域被覆性質を持つモデルの終拡張の和として実現されます.
もう少し具体的に述べれば,次のような戦略で構成が行われます:
\begin{enumerate}
 \item 任意の$r$に対し$W \subseteq W_r \subseteq V$となるような,$V$の基礎モデル$W$が取れればよい.
       そうすれば$W \subseteq W_r \subseteq V$より事実~\ref{lem:interm-ext}から$W$は各$W_r$の基礎モデルとなる.
 \item 各$W_r$が$\mathbb{P}_r$-生成拡大に関する$V$の基礎モデルであるとして,$|X|, |\mathbb{P}_r| < \kappa$が任意の$r$に対して成り立つような$\kappa$を持ってくる.
 \item すると補題~\ref{lem:ground-is-pseudoground}より各$W_r$は$V$に対し$\kappa$-被覆性質と$\kappa$-近似性質を持つ.
 \item\label{item:glob-cov-submod-many} これを使って,十分大きな任意の$\theta$に対して,$\ZFC^{\leq\mu}$の推移的モデル$M^\theta \subseteq H_\theta$で$\kappa^+$-大域被覆性質を持ち$M \subseteq \bigcap_{r \in X} W_r$となるものを取ってこれる.
 \item $\mu \defeq 2^{\kappa+}$とおけば,一意性補題~\ref{lem:global-covering-unique}より各$M^\theta$は$\power{<\mu}{2} \cap M^\theta$の値により一意に確定する.
 \item $\theta$の候補が真のクラス個あるのに対して,$\power{<\mu}{2} \cap M^\theta$の候補は集合個しかないので,任意の$\theta$に対し$M^\theta \cap \power{<\mu}{2} = p$が成り立つような$p \subseteq \power{<\mu}{2}$が取れているとしてよい.
 \item すると,上の一意性補題~\ref{lem:global-covering-unique}から$\theta < \theta'$なら$M^{\theta'} \cap H_\theta = M^\theta$となる.
       即ち,$M^{\theta'}$は$M^\theta$の終拡張(end-extension)になっている.
 \item $W \defeq \bigcup_\theta M^\theta$とおけば,終拡張の和であることから$W$は順序数を全て含み,$\ZFC$のモデルとなる.
 \item 更に$W$は明らかに$V$に対する$\kappa^+$-大域被覆性質も満たしている.
 \item よってBukovsk\'{y}の定理\ref{thm:bukovsky}より望み通り$V$は$W$の生成拡大になっている.
\end{enumerate}

以下,記事を通して以下の記法を固定しておく.

\begin{notation}
 $X$を集合とし,$\Set{W_r | r \in X}$を$V$の基礎モデルの族とする.
 各$r \in X$に対し,$V$は$W_r$の$\mathbb{P}_r$-生成拡大だとして,$|X|, |\mathbb{P}_r| < \kappa$となる正則基数$\kappa$を一つ固定する.
 このとき,$W_r$は補題~\ref{lem:ground-is-pseudoground}および定理~\ref{thm:bukovsky}より$\kappa$-大域被覆性質と$\kappa$-近似性質を持つ.
 $\mu \defeq (2^{\kappa})^{+}$とする.
\end{notation}

\subsection{$\kappa^+$-大域被覆がいっぱい}
では上の方針に沿って証明をしていこう.
まずは上の~\ref{item:glob-cov-submod-many}までの部分を示す.
\begin{lemma}\label{lem:cof-many-gl-cov}
 任意の$\alpha$に対し,正則基数$\theta > \kappa$と推移的な$\ZFC^{\leq\mu}$のモデル$M^\theta \subseteq H_\theta$で$(M, H_\theta)$が$\kappa^+$-大域被覆性質を持ち$M \subseteq \bigcap_{r \in X} W_r$となるものが取れる.
\end{lemma}
\begin{proof}
 $H_\theta \models \ZFC^{\leq\mu}$を満たす正則基数$\theta$は非有界に存在するので,一つそんな$\theta$を固定する.
 $\gamma \defeq (\theta^{<\theta})^V$とおいて,$V$における$\power{<\theta}{\theta}$を$\Set{f_i | i < \gamma}$により列挙し,$h: \theta \times \gamma \to \theta$を次で定める:
 \[
 h(\alpha, i) \defeq
 \begin{cases}
  f_i(\alpha) & (\text{if } \alpha \in \dom(f_i))\\
  0             & (\ow).
 \end{cases}
 \]
 この$f_i$達は$H_\theta$において$\kappa^+$-大域被覆性質の適用対象となるような$f: \alpha \to \On$, $f \in H_\theta$を列挙していて,$h$は一本でその辞書の役割をしていると思える.
 \begin{claim}
  $H: \theta \times \gamma \to [\theta]^{\leq \kappa}$で$H \in \bigcap_r W_r$かつ$h(i, \alpha) \in H(i, \alpha)$を満たすものが存在.
 \end{claim}
 この$H$から各$f_i: \alpha \to \theta \in H_\theta$を被覆する関数が得られる.
 \begin{subproof}
  $\eta < \kappa$に関する帰納法で,次を満たすように各$r \in X$に対し$H_{\eta, r}$を取っていく:
  \begin{enumerate}
   \item $H_{\eta, r} \in W_r$, $H_{\eta, r}: \theta \times \gamma \to [\theta]^{<\kappa}$,
   \item $\displaystyle
            H_{\eta,r}(\alpha, i)
           \supseteq \set{h(\alpha, i)} \cup \bigcup_{{s \in X}\atop{\zeta < \eta}}H_{\zeta, s}(\alpha, i)$.
  \end{enumerate}
  いま各$W_r$が$V$に対して$\kappa$-大域被覆性質を持つことに注意する.

  任意の$\zeta < \eta$と$s \in X$に対し$H_{\zeta, s}$が定義出来ているとする.
  この時$V$で
  \[
   H'(\alpha, i) \defeq \set{h(\alpha, i)} \cup \bigcup_{s \in X, \zeta < \eta} H_{\zeta, s}(\alpha, i)
  \]
  とおくと,$|X \times \eta| < \kappa$および$|H_{\zeta, s}(\alpha, i)| < \kappa$より$|H'(\alpha, i)| < \kappa$である.
  よって$W_r$における$\kappa$-大域被覆性質から$W \models H_{\eta, r} : \theta \times \gamma \to [\theta]^{<\kappa}$を満たす$H_{\eta, r} \in W$で$H'(\alpha, i) \subseteq H_{\eta, r}(\alpha, i)$となるものが取れる(これは$(\alpha, i) \mapsto \set{H'(\alpha, i)}$を考えて,その候補を$\kappa$個未満で近似するものを取り,値域の和を取ればよい).
  これが求めるものである.

  さて,条件を満たす$H_{\eta, r}$が取れた.
  そこで$r_0 \in X$を適当に固定して,$H(\alpha, i) \defeq \bigcup_{\eta < \kappa} H_{\eta, r_0}(\alpha, i)$とおく.
  これが求める$H$であることを示す.
  特に$H \in \bigcap_{r\in X} W_r$が言えればよい.

  とくにいま,各$W_r$は$V$に対して$\kappa$-近似性質を持つから,
  \[
   E \defeq \Set{ (\alpha, i, \xi) | \eta \in H(\alpha, i)}
  \]
  とおいて,任意の$z \in [\theta \times \gamma \times \theta]^{<\kappa} \cap W_r$に対して$z \cap E \in W_r$となることが示せればよい.
  $d \defeq \Set{ (\alpha, i) | \exists \xi \: (\alpha, i, \xi) \in z}$とおけば$|d| < \kappa$であり,各$(\alpha, i) \in d$に対し$\Set{\xi < \theta | (\alpha, i, \xi) \in E \cap z}$も濃度$\kappa$未満である.
  よって$\kappa$の正則性から,各$(\alpha, i)$に対し$\eta(\alpha, i) < \kappa$で$\Set{\xi < \theta | (\alpha, \eta, \xi) \in E \cap z} \subseteq H_{\eta(\alpha, i), r}(\alpha, i)$となるものが取れる.
  ふたたび$\kappa$の正則性から,$\eta^* \defeq \sup_{(\alpha, i) \in d} \eta(\alpha, i) < \kappa$となるので,
  \[
   E \cap z = \Set{ (\alpha, i, \xi) \in z | \xi \in H_{\eta^*, r}(\alpha, i)} \in W_r.
  \]
  $z$は任意なので,$E \in W_r$を得る.
  これが示したかったことである. \qed
 \end{subproof}
 このとき,全単射$\pi: \theta \times \gamma \times \theta \congto \gamma$で$\pi \in L$を満たすものを取り,$A \defeq p[H]$とおく.
 $A$と$H$は互いに同じ情報を持っているから,明らかに$A \in \bigcap_r W_r$であり,また$A$は順序数の集合なので,$L[A]\models \ZFC$である.
 そこで$M^\theta \defeq H_\theta \cap L[A]$とおく.
 $L[A]$は$L$と$A$を含む最小のモデルだったから,$\theta \subseteq M^\theta \subseteq \bigcap_r W_r$となっている.

 まず$M^\theta$が$\kappa^+$-大域被覆性質を満たすことを見よう.
 ここで,$H_\theta$から任意に$f: \alpha \to \theta$を取ってくれば,$\vec{f}$の取り方から$f_i = f$となる$i < \gamma$が存在する.
 この時$F(\beta) \defeq H(\beta, i^*)$により$F: \alpha \to [\theta]^{\leq \kappa}$を定めれば,
 \[
  \left| \trcl(F) \right| = \alpha \cdot \sup_{\beta < \alpha} \underbrace{|F(\beta)|}_{< \theta} < \theta.
 \]
 よって$F \in H_\theta$.
 また$F$は$H$から定義出来るので$F \in L[A]$となり,結局$F \in L[A] \cap H_\theta = M$となる.
 このとき定め方から$\beta < \alpha$に対し$f(\beta) = f_i(\beta) = h(\beta, i) \in H(\beta, i) = F(\beta)$となるので,$F \in M$が$f$を被覆する関数になっている.

 あとは$M^\theta \models \ZFC^{\leq\mu}$が示せればよい.
 実は,$M$は$L[A]$における$H_\theta$そのものと一致していて,特に$\theta$の取り方から$M \models \ZFC^{\leq \mu}$となっている.
 実際, $x \in L[H]$が$V$において$\left\lvert\trcl(\set{x})\right\rvert < \theta$を満たすなら,$V$において$\alpha < \theta$と$\varphi: \alpha \twoheadrightarrow x$が取れるが,$M^\theta$が$H_\theta^V$において$\kappa^+$-大域被覆性質を持つことから,$F: \alpha \to [M^\theta]^{\kappa}$で$F \in M^\theta$かつ$f(\beta) \in F(\beta)$を満たすものがとれる.
 特に$x \subseteq \ran(F)$なので$M^\theta$においても$|\trcl({x})| \leq \bigcup \ran(F) \leq \kappa \cdot \alpha < \theta$となり,結局$L[H] \models \left\lvert\set{x}\right\rvert < \theta$となり,$x \in H_\theta^{L[H]}$を得る.

 以上より求める$M^\theta$の存在が示せた. \qed
\end{proof}

\subsection{小さな大域被覆モデルを貼り合わせる}
前節で~\ref{item:glob-cov-submod-many}までは終わった.
残りは割合すんなりと行く.まず次の二つの事実は標準的なので認める:
\begin{fact}\label{lem:ZF-model-char}
 順序数を全て含む推移的クラス$W \subseteq V$について次は同値:
 \begin{enumerate}
  \item $W \models \ZF$,
  \item $W$はG\"{o}del演算について閉じ,概宇宙的(i.e.\ $\forall z \subseteq W: \text{set}\: \exists y \in W\: z \subseteq y$).
        \label{}
 \end{enumerate}
\end{fact}

さて,$\mu = 2^{\kappa+}$と略記していたことを思い出そう.
簡単な観察で次がわかる:
\begin{lemma}
 $p^* \subseteq \power{<\mu}{2}$で,$M^\theta \cap \power{<\mu}{2} = p^*$となる$\theta$が非有界に存在するものが存在する.
\end{lemma}
\begin{proof}
 補題~\ref{lem:cof-many-gl-cov}より各$M^\theta$は$H_\theta$に対し$\kappa^+$-被覆性質を持つ$\ZFC^{\leq\mu}$のモデルである.
 $p^\theta = \power{<\mu}{2} \cap M^\theta$とおけば,ukovsk\'{y}の定理の系として得られる補題~\ref{lem:global-covering-unique}より,$M^\theta$は$\power{<\mu}{2} \cap M = p^*$かつ$M \models \ZFC^{\leq\mu}$となる一意な$H_\theta$の内部モデルとなるのだった.

 いま$\theta$は非有界に(つまり真のクラス個)存在するのに対し,$p^\theta$の候補は高々$2^{2^{<\mu}}$個しか存在しない.
 よって鳩ノ巣原理から$p^* \subseteq \power{<\mu}{2}$で$p^* = p^\theta$となる$\theta$が非有界に存在するものが少なくとも一つは取れる. \qed
\end{proof}

\begin{notation}
 上の補題の$p^*$を固定し,$I \defeq \Set{ \theta | \exists M^\theta \: M^\theta \cap \power{<\mu}{2} = p^* }$と置く.

 $W \defeq \bigcup_{\theta \in I} M^\theta$と定める.
\end{notation}

このとき,$\Braket{M^\theta | \theta \in I}$は終拡張列になっていることがわかります:
\begin{lemma}
 $\theta, \theta' \in I$, $\theta < \theta'$とするとき,$M^{\theta'} \cap H_\theta = M^\theta$.
\end{lemma}
\begin{proof}
 $\theta < \theta'$とし,$M' \defeq M^{\theta'} \cap H_\theta$とおく.
 この時,$M'$は$H_\theta$に対し$\kappa^+$-大域被覆性質を持つ.
 実際,$\alpha < \theta$, $f: \alpha \to \theta$とすると,$M^{\theta'}$の$\kappa^+$-大域被覆性質から$F: \alpha \to [\theta]^{\kappa}$で$F \in M'$を満たすものが取れる.
 しかし,$\alpha, \kappa < \theta$より実際には$F \in H_\theta$となっているので,$F \in M'$を得る.

 一方,明らかに$M' \cap \power{<\mu}{2} = M^{\theta'} \cap \power{<\mu}{2} = p^*$であるが,$I$の定義より$M^\theta \cap \power{<\mu}{2} = p^*$でもあるため,一意性補題~\ref{lem:global-covering-unique}から$M' = M^\theta$となる. \qed
\end{proof}

\begin{lemma}
 $W$は全ての順序数を含む推移的モデルで$W \subseteq \bigcap_{r \in X} W_r$であり,$W \models \ZFC$.
\end{lemma}
\begin{proof}
 $I$の非有界性より$W$が全ての順序数を含むのは明らか.
 また推移的モデルの和なので$W$自身推移的である.
 また,取り方から$M^\theta \subseteq \bigcap_r W_r$なので,その和も当然$\bigcap_r W_r$に含まれる.

 $W \models \ZF$を示そう.
 特に事実\ref{lem:ZF-model-char}を使うことを考える.
 G\"{o}del演算が何なのかここでは指定していないが,有限個の基本的な集合演算であり,各$M^\theta$は全てそれらで閉じている.
 そうした集合の増加列の和なので,結局$W$はG\"{o}del演算で閉じていることは明らかである.
 概宇宙性を示そう.
 いま$z \subseteq W$が集合なら,$I$の非有界性より$z \subseteq H_\theta$となる$\theta \in I$が存在する.
 このとき$z \subseteq W \cap H_\theta = M^\theta$となるが,$M^\theta = H_\theta \cap W = H_\theta^W \in W$なので問題ない.

 最後に$W \models \AC$だが,いつものように整列定理を示す.
 しかし$x \in W$とすれば,$x \in M^\theta \models \ZFC^{\leq\mu}$より$x$の整列順序$w \in M^\theta$が存在する.
 これは明らかに$W$にも入るので,$W$は整列定理を満たす. \qed
\end{proof}

ここまでくれば,あとは一瞬である.

\begin{proof}[sDDGの証明]
 あと残っていることは任意の$r \in X$に対し$W$が$W_r$の基礎モデルとなることを示すだけである.
 中間拡大補題~\ref{lem:interm-ext}を念頭におくと,$W$が$V$の基礎モデルとなっていることを示せれば,間に挟まれた$W_r$は$W$の生成拡大になっていることが言える.
 これには,Bukovsk\'{y}の定理~\ref{thm:bukovsky}より$(W, V)$が大域被覆性質を持つことが言えればよい.
 実際,$(W, V)$は$\kappa^+$-大域被覆性質を持つ.
 任意に$f: \alpha \to \On$を取る.
 $\theta > \alpha$となる$\theta \in I$を取れば,$f \in H_\theta$である.
 いま$(M^\theta, H_\theta)$は$\kappa^+$-大域被覆性質を持つので,$F \in M^\theta \subseteq W$で$F: \alpha \to [\theta]^{\kappa}$かつ$f(\alpha) \in F(\alpha)$となるものが取れる.
 $W$は$M^\theta$たちの和だったから,特に$F \in W$となる.

 よって$(W, V)$は$\kappa^+$-大域被覆性質を持ち,特に$V$は$W$の$\kappa^+$-\cc{}生成拡大となっているから,間に挟まれた$W_r$たちは$W$の生成拡大になっている. \qed
\end{proof}
\nocite{Fuchs:2014fj,Usuba:2017fp}
\nocite{Friedman:2016lr}
\printbibliography[title=参考文献]
\end{document}
