---
title: 集合論の地質学3:Bukovskýの定理──強制拡大の特徴付け
author: 石井大海
description: |
  集合論の地質学に関する記事の第三回目.
  最終目標の下方有向性原理の証明の準備として,Bukovskýによる強制拡大の特徴付けを証明します.
  [前回はこちら](/math/geology-mantle-and-ddg.html).
  [初回はこちら](/math/geology-ground-definability.html).
latexmk: -pdflua
date: 2017/11/30 00:20:34 JST
tag: Boole値モデル,無矛盾性証明,強制法,公理的集合論,集合論,生成多元宇宙,集合論の地質学,Bukovskýの定理
macros:
  infdisj: |
    \mathop{\mathchoice{
      \bigvee\hspace{-2ex}\bigvee
    }{\bigvee\hspace{-1.6ex}\bigvee
    }{\bigvee\hspace{-1.4ex}\bigvee
    }{\bigvee\hspace{-1ex}\bigvee
    }}
  Col: '\mathop{\mathrm{Col}}'
  truth:
    argNums: 2
    body: |
      {\mathchoice{%
        \left[\!\!\left[#2\right]\!\!\right]
      }{\left[\!\!\left[#2\right]\!\!\right]
      }{\left[\!\!\left[#2\right]\!\!\right]
      }{\left[\!\!\left[#2\right]\!\!\right]}}_{#1}
    optArgs: ['']
---
\documentclass[a4j,leqno]{ltjsarticle}
\usepackage[hiragino-pron]{luatexja-preset}
\usepackage{mystyle}
\usepackage{luatexja-otf}
\usepackage{dsfont}
\DeclareMathAlphabet{\mathrsfs}{U}{rsfso}{m}{n}
\renewcommand{\mathscr}[1]{\mathup{\mathrsfs{#1}}}
\usepackage{fixme}
\usepackage[super]{nth}
\usepackage[bookmarksnumbered,pdfproducer={LuaLaTeX},%
            luatex,psdextra,pdfusetitle,pdfencoding=auto]{hyperref}
\usepackage[backend=biber, style=math-numeric]{biblatex}
\addbibresource{myreference.bib}
\renewcommand{\emph}[1]{\textgt{\textsf{#1}}}
\usetikzlibrary{shapes,shapes.geometric,positioning,math,fpu}

\title{集合論の地質学3:Bukovsk\'{y}の定理──強制拡大の特徴付け}
\newcommand{\mantle}{\mathbb{M}}
\newcommand{\cc}{c.c.\ }
\newcommand{\M}{\mantle}
\newcommand{\gM}{g\mathbb{M}}
\newcommand{\gmantle}{\gM}
\newcommand{\DDG}{\mathord{\mathrm{DDG}}}
\newcommand{\sDDG}{\mathord{\mathrm{sDDG}}}
\author{石井大海}

\usepackage{amssymb}	% required for `\mathbb' (yatex added)
\usepackage{amsmath}	% required for `\gather*' (yatex added)
\begin{document}
\maketitle

\begin{abstract}
 本稿は集合論の地質学に関する記事の第三回目です:
 \begin{enumerate}[label={\arabic*.}]
  \item \href{http://konn-san.com/math/geology-ground-definability.html}{概観と基礎モデルの定義可能性}
  \item \href{http://konn-san.com/math/geology-mantle-and-ddg.html}{マントルの構造と下方有向性原理}
  \item Bukovsk\'{y}の定理──強制拡大の特徴付け(今回)
  \item \href{http://konn-san.com/math/geology-proof-of-ddg.html}{下方有向性原理の証明}
 \end{enumerate} 
 \emph{集合論の地質学}は,与えられた集合論の宇宙$V$の内部モデルがいかなる生成拡大になっているかを考える集合論の分野です.
 薄葉~\cite{Usuba:2017fp}を種本にその基本定理である\emph{下方有向性原理}を示すのがこのシリーズの目標ですが,今回はその中で大きな役割を果す,\emph{Bukovsk\'{y}による強制拡大の特徴付け}を証明します.
\end{abstract}

\section{Bukovsk\'{y}の定理:大域被覆性質による特徴付け}
前回までの結果により,与えられた宇宙$V$の基礎モデルすべてを一様に列挙出来るようになり,マントルという構造を定義出来るようになった.
\begin{definition}
 \begin{itemize}
  \item \emph{マントル}$\M$とは$V$の基礎モデルすべての共通部分である.
  \item \emph{生成マントル}$\gM$とは$V$のすべての生成拡大のマントルの共通部分である.
 \end{itemize}
\end{definition}
$\gM$と$\M$は一致するのか?これらは$\ZFC$のモデルになるのか?といった問題が自然と浮かんでくるが,そうした問題は薄葉~\cite{Usuba:2017fp}によって証明された次の\emph{基礎モデルの強い下方有向性定理}によって一挙に解決されたのだった:
\begin{theorem}[強い下方有向性定理,$\sDDG$;Usuba~\cite{Usuba:2017fp}]
 任意の集合$X$に対し,$\Set{ W_r | r \in X }$の共通の基礎モデルが存在する.
 但し,$W_r$は$r$によって決定される$V$の基礎モデルを表す.
\end{theorem}
\begin{corollary}
 $\M = \gM$であり,$\M$は$\ZFC$を満たす内部モデル.
\end{corollary}

いよいよこの定理を示していきたいが,その中では次の古典的なBukovsk\'{y}の定理が重要な役割を果す:
\begin{definition}
 $M \subseteq V$を共に$\ZFC$のモデルとする.
 $(M, V)$が\emph{$\kappa$-大域被覆性質}($\kappa$-\emph{global covering property}; $\kappa$-GCP)を持つ$\defs$任意の順序数$\alpha \in \On$と$V$の関数$f: \alpha \to \On$に対して,$F: \alpha \to [\On]^{<\kappa}$で$F \in M$かつ$f(\alpha) \in F(\alpha)$を満たすものが取れる.
\end{definition}
\begin{theorem}[Bukovsk\'{y}\cite{Bukovsky:1973ph}]
 $M \subseteq V$を$\ZFC$の内部モデルとし,$\kappa$が$M$で正則基数であるとする.
 このとき,次は同値:
 \begin{enumerate}
  \item $\kappa$-c.c.擬順序$\mathbb{P} \in M$と$(M, \mathbb{P})$-生成フィルター$G \in V$があって$V = M[G]$.
  \item $(M, V)$は$\kappa$-大域被覆性を持つ.
 \end{enumerate}
\end{theorem}
任意の擬順序集合$\mathbb{P}$について,$\mathbb{P}$は$|\mathbb{P}|$-\cc{}を持つので,上のBukovsk\'{y}の定理はあらゆる強制法\footnote{正確には,擬順序\emph{集合}による強制法の特徴付けであり,真のクラスを成すような擬順序による強制法の特徴付けにはならない.本稿では断わらない限り擬順序集合による強制法しか考えない.}による強制拡大の特徴付けを与えている.

そこで,今回はFriedman--Fuchino--Sakai~\cite{Friedman:2016lr}とSchindler~\cite{Schindler:2017eu}を参考にBukovsk\'{y}の定理の現代的な証明を与える.

強制法の一般論から,$\kappa$-c.c.\ 拡大なら$\kappa$-大域被覆性質を持つことはわかる:
\begin{lemma}
 $\mathbb{P}$が$\kappa$-c.c.\ 強制概念,$G$が$(V, \mathbb{P})$-生成フィルターの時,$(V, V[G])$は$\kappa$-大域被覆性質を持つ.
\end{lemma}
\begin{proof}
 $V[G]$において$f: \alpha \to \On$を固定し,$\dot{f}^G = f$となる$\mathbb{P}$-名称$\dot{f} \in V^{\mathbb{P}}$をとる.
 このとき,各$\xi < \alpha$に対し$\mathcal{A}_\xi$を$\Set{p \in \mathbb{P} | \exists \beta \: p \Vdash \quoted{\dot{f}(\check{\xi}) = \check{\beta}}}$に含まれる中で極大な反鎖とする.
 $\mathbb{P}$は$\kappa$-c.c.\ を満たすので,各$\xi$につき$|\mathcal{A}_\xi| < \kappa$となる.
 そこで$F(\xi) \defeq \Set{ \beta | \exists p \in \mathcal{A}_\xi \: p \Vdash \quoted{\dot{f}(\check{\xi}) = \check{\beta}} }$とおけば,各$F(\xi)$の候補は高々$\kappa$個未満しかないので,$F: \alpha \to [\On]^{<\kappa}$かつ$F \in V$となる.
 また,定義より明らかに$f(\xi) \in F(\xi)$が成り立つ. \qed
\end{proof}

\subsection{大域被覆性質から$\kappa$-\cc{}強制拡大の特徴付け}
よって,後は逆向きの命題,即ち「$(W, V)$が$\kappa$-大域被覆性質を持つなら$V$は$W$の$\kappa$-\cc{}生成拡大となる」を示していけばよい.
つまり,$(W, V)$が$\kappa$-大域被覆性質を満たすとして,$W$において$\kappa$-\cc{}を満たす$\mathbb{P} \in W$と$(W, \mathbb{P})$-生成的な$G \in V$で$V = W[G]$を満たすようなものを見付けてこなくてはいけない.

これは次のような戦略で示される:

\begin{enumerate}
 \item\label{item:global-covering-to-k-cc} $(W, V)$が$\kappa$-大域被覆性質を持つなら,$V$の任意の集合は$\kappa$-\cc{}強制法で$W$に付け加えることが出来る.
 \item\label{item:nontriv-adds-2-kappa} また,非自明な$\kappa$-\cc{}強制法は必ず$2^{<\kappa}$の部分集合を付け加える.
 \begin{itemize}
  \item つまり,$\Pow(2^{<\kappa})$を変えない$\kappa$-\cc{}強制法は自明で,宇宙に何の影響を与えないことになる.
 \end{itemize}
 \item そこで$\kappa$-\cc{}強制法で$A = \Pow^{V}(2^{<\kappa})$を$W$に付け加え$W[A]$とする.
       $W[A] = V$となっていることを示せばよい.
 \item このとき$(W[A], V)$も$\kappa$-大域被覆性質を持ち,$\kappa$-\cc{}拡大によって$V$の任意の集合$x \in V$を付け加えられる.
 \item しかし,$W[A]$の時点で$\Pow(2^{<\kappa})$はめいっぱいに取れているので,$W[A]$と$W[A][x]$で$\Pow(2^{<\kappa})$の値は一致する.
 \item よって$x \in W[A][x] = W[A]$となり,$x \in V$は任意だったから$V = W[A]$となる.
\end{enumerate}

\subsection{非自明な$\kappa$-\cc{}強制法は$2^{<\kappa}$の部分集合を足す}
まずは上の~\ref{item:nontriv-adds-2-kappa}を示そう.
これは次のようにして示せる:
\begin{lemma}\label{lem:nontriv-2<kappa}
 $\mathbb{P}$が非自明な$\kappa$-\cc{}強制概念なら$\mathbb{P}$は新たな$2^{<\kappa}$の部分集合を足す.
\end{lemma}
\begin{proof}
 $\dot{T}$を$V$に属さない順序数の集合の名前とする:$\Vdash_{\mathbb{P}} \quoted{\dot{T} \in \Pow(\On) \setminus \check{V}}$.
 十分大きな$\theta$をとり,次を満たす$M \prec \mathcal{H}_\theta$をとる:
 \begin{enumerate}
 	\item $|M| \leq 2^{<\kappa}$,
 	\item $\power{<\kappa}{M} \subseteq M$,
 	\item $\dot{T}, \mathbb{P}, \kappa \in M$.
 \end{enumerate}
 $\dot{S}$を${}\Vdash_{\mathbb{P}} \dot{S} = \dot{T} \cap \check{M}$なる$\mathbb{P}$-名称とする.
 このとき$\Vdash_{\mathbb{P}} \quoted{\dot{S} \notin \check{V}}$が示せれば良い.
 そこで$p \in \mathbb{P}$と$S \in \Pow(2^{<\kappa}) \cap V$で$p \Vdash \quoted{\dot{S} = \check{S}}$となるものがあったとして矛盾を導く.
 簡単のため$\mathbb{P}$は完備Boole代数だとして,$p \defeq \truth{\dot{S} = \check{S}} > 0$となら$\mathbb{P}$が濃度$\kappa$の反鎖を持つことを示せばよい(\emph{背理法}).

 発想としては,$\dot{S}$の値が$\check{S}$と途中の桁までは一致するが,ある点で食い違うように強制する条件の列が求めるものになる.
 具体的には,以下を満たすように$\xi_\alpha \in M \cap \On$および$p_\alpha, q_\alpha \in \mathbb{P} \cap M$をとっていく:
 \begin{enumerate}
  \item $0 < p_\beta, q_\beta \leq p_\alpha$ and $p \leq p_\alpha$ if $\alpha < \beta < \kappa$,
  \item $\xi_\alpha \in S$なら$p_\alpha \Vdash \check{\xi}_\alpha \in \dot{T}$ and $q_\alpha \Vdash \check{\xi} \notin \dot{T}$.
        $\xi_\alpha \notin S$なら$p_\alpha$と$q_\alpha$の役割を逆にする.
 \end{enumerate}
 \parpic[r]{\begin{tikzpicture}
 \node (A) [dashed,draw,rotate=60,ellipse,label={above right:$\dot{A}$},minimum width=6cm,minimum height=3cm] {};
 \node (M) [anchor=north west,fill=none,draw,below left=8mm and -8mm of A.south east,ellipse,minimum height=3.8cm,minimum width=5cm,label={below right:$M$}] {};
 \begin{scope}
   \clip (M) ellipse (2.5cm and 1.9cm);
   \fill [rotate=60,fill=gray,fill opacity=.5] (A) ellipse (3cm and 15mm);
 \end{scope}
 \node [above=1mm of M] {$S$};
 \begin{scope}[sibling distance=1.6cm,level distance=1cm]
  \node (x0l) at ($(A)!.5!(M) + (1cm,1cm)$) {$\xi_0$}
      child {
        node (x1l) {$\xi_1$}
          child[very thick,densely dotted] {
              node (xal) {$\xi_\alpha$}
                    child[densely dotted,very thick] {node  {$\phantom{\xi_\alpha}$} }
                    child[thin,solid,level distance=8mm] {node (xar) {$\xi_\alpha$}}
          }
          child { node (x1r) {$\xi_1$} }
      }
      child { 
        node (x0r) {$\xi_0$}
      };
      \foreach \lab/\l/\r in {0/x0l/x0r,1/x1l/x1r,\alpha/xal/xar} {
       \node[left=0mm of \l,anchor=east,inner sep=0pt] {$p_{\lab} \Vdash$};
       \node[right=0mm of \r,anchor=west, inner sep=0pt] {$\mathrel{\reflectbox{$\Vdash$}} q_{\lab}$};
      }
 \end{scope}
\end{tikzpicture}}
 こうすれば,$\mathcal{A} = \Set{q_\alpha | \alpha < \kappa}$が濃度$\kappa$の反鎖となるのは明らかである.

 では実際にとっていく.
 $\gamma < \kappa$に対して,$\Braket{p_\alpha, q_\beta | \beta < \gamma}$まで上記を満たすように取れているとする.
 $M$が${<}\kappa$-列で閉じていることから$\Braket{p_\alpha | \alpha < \gamma } \in M$となり,特に$p' \defeq \prod_{\alpha < \gamma} p_\alpha \in M$である.
 特に,$p_\alpha$たちは$\check{S}$を近似するようにとれているから,$p = \truth{\dot{S} = \check{S}}$より$0 < p \leq p'$となる事に注意する.
 一方,初等性より$M \models \quoted{{} \Vdash_{\mathbb{P}} \dot{T} \notin \check{V}}$なので,$\xi_\gamma \in M \cap \On$で$p' \cdot \truth{\xi_\gamma \in \dot{T}}, p' \cdot \truth{\xi_\gamma \notin \dot{T}}$が共に正となるものが取れる.
 そこで$\xi_\gamma \in S$なら$p_\gamma \defeq p' \cdot \truth{\xi_\gamma \in \dot{T}}$, $q_\gamma \defeq p' \cdot \truth{\xi_\gamma \notin \dot{T}}$とし,そうでなければ逆になるように$p_\gamma, q_\gamma$を定めれば,上記の条件を満たすように取れる.

 よって望み通り$\mathcal{A} = \Set{q_\alpha | \alpha < \kappa}$が取れた.
 ここでもし$\alpha < \beta$なら$q_\alpha$と$q_\beta$は$\xi_\alpha$の$\dot{T}$への所属の可否で互いに食い違っているので両立しないから,$\mathcal{A}$は濃度$\kappa$の$\mathbb{P}$の反鎖となる. \qed
\end{proof}

\begin{corollary}\label{cor:2<kappa-cc-tirival}
 $\mathbb{P}$が$\kappa$-\cc{}強制概念,$G$が$(V, \mathbb{P})$-生成的で$\Pow(2^{<\kappa}) \cap V = \Pow(2^{<\kappa}) \cap V[G]$なら,$V = V[G]$.
\end{corollary}

\subsection{大域被覆性から$\kappa$-\cc{}強制法を得る}
よって,後は上の~\ref{item:global-covering-to-k-cc},つまり$(W, V)$が$\kappa$-大域被覆性質を持つなら,$V$の任意の元を$\kappa$-\cc{}強制法で付け加えられる事を示そう.
より強く,任意の$x \in V$が何らかの$\kappa$-\cc{}強制法$\mathbb{P}_x \in W$の生成フィルター$G_x$と一対一に対応しており,$V[G_x]$が$x$を含む最小の$V$の拡大となることを示す.

いま,我々が考えているのは$\ZFC$のモデルだけであり,任意の集合は順序数の集合でコードできる.
よって,特に$V$の順序数の集合についてだけ考えれば良い.

そこで,ある特定の順序数の部分集合の情報を記述するのに十分な無限論理の体系を考えよう.
\begin{definition}
 $\kappa \leq \mu$を無限基数とする.
 無限論理$\mathcal{L} \defeq \mathcal{L}_\kappa(\mu)$を次で定める:
 \begin{description}
  \item[述語記号] 各$\xi < \mu$に対し$\check{\xi} \in \dot{a}$をゼロ項述語記号(述語定数)とする.
  \item[論理式]
  \begin{enumerate}
   \item 原子論理式$\quoted{\check{\xi} \in \dot{a}}$は論理式である.
   \item $\varphi$が論理式なら$\neg \varphi$も論理式.
   \item $\Gamma \subseteq \mathcal{L}$が$\kappa$個未満の論理式の集合なら,$\infdisj \Gamma$も論理式.
  \end{enumerate}
 \end{description}
\end{definition}
\begin{remark}
 \begin{itemize}
  \item $\mathcal{L}_\kappa(\mu)$の論理式の集合は,考えるモデルによって変わる.
 実際,$W$の中では濃度$\kappa$以上だった集合$\Gamma$が$V$の中では$\kappa$未満になっている場合があり,この場合$\infdisj \Gamma$は$W$では定義されないが$V$では定義される.
  \item $\mathcal{L}_{\kappa}(\mu)$の論理式の個数は$\mu^{<\kappa}$個である.
 \end{itemize}
\end{remark}
適当な集合$A \subseteq \mu$に対して,自然に$\mathcal{L}_\kappa(\mu)$-構造が定まる:
\begin{definition}
 $\varphi$を$\mathcal{L}_\kappa(\mu)$-論理式,$A \subseteq \mu$とする.
 充足関係$A \models \varphi$を次で定める:
 \begin{itemize}
  \item $A \models \quoted{\check{\xi} \in \dot{a}} \defs \xi \in A$,
  \item $A \models \neg \varphi \defs A \nvDash \varphi$,
  \item $A \models \infdisj \Gamma \defs \exists \varphi \in \Gamma \: A \models \varphi$.
 \end{itemize}
 また,公理系$\Gamma \subseteq \mathcal{L}_{\kappa}(\mu)$に対し,$A \models \Gamma$は$\forall \varphi \in \Gamma \: A \models \varphi$のこととする.
\end{definition}

以上で無限論理の意味論は与えたので,今度は証明体系も与えたい.
無限長のオブジェクトを扱っているので,何処で考えるかによって証明体系は変わってきそうである.
つまり,$W$を含む(推移的)モデルの間で,$\quoted{\Gamma \vdash \varphi}$の意味がかわらないようにしたい.
だって,$W$の中で「$\varphi$は$\Gamma$の定理だよ!」と言うのに,より広い$V$の中で「定理じゃないよ!」ということになっていては困る.
これには,実際に無限長の論理式と証明木を使って定義する方法もある(Friedman--Fuchino--Sakai~\cite{Friedman:2016lr}を参照)が,今回は強制法と記述集合論の結果を使ってやることにする.

その中心的な役割を担うのが,次の崩壊強制法である:
\begin{definition}
 順序数$\alpha \geq \omega$を可算に潰す崩壊強制$\Col(\omega, \alpha)$とは$\finseq{\alpha}$を台集合とし,$p \leq q \defs p \supseteq q$により順序を定めたものである.
\end{definition}
\begin{remark}
 \begin{itemize}
  \item 「有限列」の概念はどの推移的モデルで考えても変わらないので,任意の順序数$\alpha$に対し$\Col(\omega, \alpha)$は常に一致する.
  \item $G$を$\Col(\omega, \alpha)$-生成的とするとき,$f \defeq \bigcup G$とおけば,$D_\xi = \Set{p | \xi \in \ran(p)}\;(\xi < \alpha)$の稠密性より$f$は$\omega$から$\alpha$への全射となり,$V[G] \models |\alpha| = \aleph_0$となる.
 \end{itemize}
\end{remark}
なぜ可算に潰すような強制法を考えるのかといえば,崩壊強制法により$W$における$\mathcal{L}$-論理式の全体を可算にしてしまえば,各論理式$\varphi \in \mathcal{L}_{\kappa}^W(\mu)$はあるBorel集合と同一視出来るからである.

なぜか?
まず$\nu = (\mu^{<\kappa})^{W}$が可算になっているので,まず$\mu$は可算集合であり,各元$\xi < \mu$は自然数$n < \omega$と一対一に対応させられる.
このとき,原子論理式$n \in \dot{a}$は,「$\dot{a}$の二進展開の$n$桁目が$1$になる」という命題だと思え,これは$\power{\omega}{2}$の基本開集合に当たる.
否定や無限選言を取る所も,集合演算としては補集合と可算和を取るところに対応している.
それで閉じているのが$\mathcal{L}_\kappa(\mu)$だから,結局$A \models \varphi$は$A$が「コード」している実数が$\varphi$の定めるBorel集合に含まれているか?という述語に他ならないのである.

では,Borel集合としてコード出来ると何が嬉しいのか?実は,実数の集合論において次の絶対性が成り立つことが知られている:
\begin{theorem}[Mostowski Absoluteness]
 $\boldface{\Pi}^1_1$-論理式は任意の推移的モデルの間で絶対的.
 即ち,$M \subseteq V$が推移的モデルだとして,$\varphi$が$M$にパラメータを持つ論理式で,Borel集合を定義するとする:$M \models \quoted{\Set{(x, z) | \varphi(x, z)}: \text{Borel}}$.
 この時,$x \in M$に対し:
 \[
  M \models \forall z \: \varphi(x, z) \iff V \models \forall z \: \varphi(x, z).
 \]
\end{theorem}
証明はたとえば拙稿「\href{http://konn-san.com/math/absoluteness-cheatsheet.html}{絶対性チートシート}」\cite{Ishii:2016db}を参照.

これを踏まえて,$W$における$\mathcal{L}$の公理系の「証明」体系を次のように定めよう:
\begin{definition}
 $\Gamma \subseteq \mathcal{L}_\kappa(\mu)$,$\varphi \in \mathcal{L}_\kappa(\mu)$とする.
 この時,証明可能性述語$\Gamma \vdash \varphi$を次で定める:
 \[
 \Gamma \vdash_{\mathcal{L}_\kappa(\mu)} \varphi \defs {}\Vdash_{\Col(\omega, \mu^{<\kappa})} \quoted{\forall A \models \Gamma \: A \models \varphi}.
 \]
 $\bot \deffml \infdisj \emptyset$を\emph{矛盾}と呼ぶ.
 $\Gamma \nvdash \bot$の時$\Gamma$は\emph{無矛盾}であるという.
\end{definition}
\begin{remark}\label{rem:consis-witness}
 $A \models \Gamma$なる$A \subseteq \mu$が存在すれば$\Gamma$は無矛盾\footnote{実際には,崩壊強制法の\emph{均質性}という性質から,$\Gamma$の無矛盾性と$V^{\Col(\omega, \mu^{<\kappa})}$において$\Gamma$のモデルが存在することが同値になる.}.
\end{remark}

すると,上のMostowski絶対性を認めれば,$\Gamma \vdash \varphi$の「絶対性」が言える:
\begin{lemma}
 $W \subseteq V$を推移的モデルとする.
 $\Gamma, \varphi \in W$をそれぞれ$W$における$\mathcal{L}_\kappa(\mu)$-理論と$\mathcal{L}_\kappa(\mu)$-論理式とすると,
 \[
 \Gamma \vdash_{\mathcal{L}_\kappa(\mu)}^W \varphi \iff \Gamma \vdash^{V}_{\mathcal{L}_\kappa(\mu)} \varphi.
 \]
\end{lemma}
\begin{proof}
 以下,$\nu^W \defeq (\mu^{<\kappa})^W$,$\nu^V \defeq (\mu^{<\kappa})^V$と表す.

 「$A \models \varphi$」はBorelなのでどんな推移的モデルで見ても変わらない.
 よって,上の定義から$\Gamma \vdash \varphi$は$\boldface{\Pi}^1_1$で書ける性質である.
 $W^{\Col(\omega, \nu^W)} \subseteq V^{\Col(\omega, \nu^V)}$に気を付ければ,Mostowski絶対性より,
 \[
  W^{\Col(\omega, \nu^W)} \models \quoted{\forall A \models \Gamma \: A \models \varphi}
 \iff
  V^{\Col(\omega, \nu^{V})} \models \quoted{\forall A \models \Gamma \: A \models \varphi}
 \]
\end{proof}
\begin{corollary}\label{cor:consis-abs}
 「公理系$\Gamma$が無矛盾」は推移的モデルの間で絶対.
\end{corollary}

さて,いよいよ$\mathcal{L}$を使って$\kappa$-\cc{}強制法を定義する.
\begin{definition}
 $g: [\mathcal{L}]^{\kappa} \to [\mathcal{L}]^{<\kappa}$が$g(\Gamma) \subseteq \Gamma$を満たすとする.
 \begin{itemize}
  \item $\varphi$が\emph{$g$-違法}($g$-illegal) $\defs$ $\exists \Gamma \in [\mathcal{L}]^\kappa \cap W \: \varphi \in \Gamma \setminus g(\Gamma)$.
  \item 理論$T^g \subseteq \mathcal{L}$は$g$-違法な論理式$\varphi$と$\varphi \in \Gamma \setminus g(\Gamma)$なる$\Gamma$に対し,次の形の論理式からなる:
        \[
         \varphi \rightarrow \infdisj g(\Gamma)\; \left(\equiv \infdisj \Set{\neg \varphi, \infdisj g(\Gamma)}\right).
        \]
  \item 強制概念$(\mathbb{P}_g, \leq)$を次で定める:
        \begin{gather*}
         \mathbb{P}_g \defeq \Set{ \varphi \in \mathcal{L} | T_g \cup \set{\varphi}: \text{無矛盾} }\\
         \varphi \leq \psi \defs T_g \cup \set{\varphi} \vdash \psi.
        \end{gather*}
 \end{itemize}
\end{definition}

まず,こうして定めた$\mathbb{P}_g$が目的通り$\kappa$-\cc{}を満たすことを見よう.
\begin{lemma}\label{lem:Pg-cc}
 $\mathbb{P}_g$は$\kappa$-\cc{}を持つ.
\end{lemma}
\begin{proof}
 $\Gamma \subseteq \left[\mathbb{P}_g\right]^\kappa$を任意に取る.
 $\varphi \in \Gamma \setminus g(\Gamma)$を一つとれば,$T_g$の定義から$\varphi \rightarrow \infdisj g(\Gamma) \in T_g$なので,順序の定義より$\varphi \leq \infdisj g(\Gamma)$となる.
 このとき,$\infdisj \emptyset$は$T_g$と「矛盾」するので,$g(\Gamma) \neq \emptyset$である.
 そこで$\psi \in g(\Gamma)$を取れば,$\psi \neq \varphi$であり$\varphi, \psi$は明らかに両立する.
 よって$\Gamma$は反鎖ではない. \qed
\end{proof}

$g$の選び方によってはどうあっても矛盾するので$\mathbb{P}_g = \emptyset$になるかもしれないし,自明な元しか持たないかもしれない.
しかし,$T_g$はある集合を$\mathbb{P}_g$の形でコード出来るかを知るための十分条件を与えてくれる:
\begin{lemma}\label{lem:GA-code-suff}
 $W \subseteq V$を$\ZFC$を満たす内部モデル, $W \models g: [\mathcal{L}]^\kappa \to [\mathcal{L}]^{<\kappa}$とし,$\mathbb{P} \defeq \mathbb{P}_g^W$とおく.
 ここで,$A \in \Pow(\mu) \cap V$が$A \models (T_g)^W$を満たすなら,
 \[
  G_A \defeq \Set{\varphi \in \mathbb{P}_g | A \models \varphi}
 \]
 は$(W, \mathbb{P})$-生成フィルターであり,$A = \Set{\xi < \mu | \quoted{\check{\xi} \in \dot{a}} \in G_A} \in W[G_A]$は$W$と$A$を含む最小の$\ZFC$の推移的モデルとなっている.
\end{lemma}
\begin{proof}
 $G_A$がフィルターとなること,$A = \Set{ \xi < \mu | \quoted{\check{\xi} \in \dot{a}} \in G_A}$となることは良い.
 また,$A$と$G_A$は互いの情報を過不足なく持っているので,$W[G_A]$が生成拡大であることさえ言えれば,強制定理から最小性も従う.

 そこで$G_A$の$W$上の生成性を示そう.
 $\mathcal{A} \in W$を$\mathbb{P}$の極大反鎖とすると,$\mathbb{P}$が$W$で$\kappa$-\cc{}を持つことから$|\mathcal{A}|^W < \kappa$となるので,$\infdisj \mathcal{A}$は$\mathcal{L}$の論理式である.
 ここで$G_A$と${\infdisj}$の定義より$G_A \cap \mathcal{A} \neq \emptyset$と$A \models \infdisj \mathcal{A}$は同値である事に注意する.
 よって,もし$G \cap \mathcal{A} = \emptyset$なら$A \models \neg \infdisj \mathcal{A}$となる.
 すると注意~\ref{rem:consis-witness}より$V$で$T_g \cup \set{\neg \infdisj \mathcal{A}}$は無矛盾となり,無矛盾性の絶対性(系~\ref{cor:consis-abs})より$W$でも無矛盾である.
 よって$G_A$と$\mathbb{P}_g$の定義から$\quoted{\neg \infdisj\mathcal{A}} \in G_A$を得る.
 すると$\mathcal{A} \cap G_A = \emptyset$より$\mathcal{A}' \defeq \mathcal{A} \cup \Set{\neg \infdisj \mathcal{A}}$は$\mathcal{A}$を真に含む反鎖となるが,これは$\mathcal{A}$の極大性に反する. \qed
\end{proof}

よって,あとはどんな$A$に対しても$A \models (T^g)^W$となるような$g \in W$が取れることが言えればよい.
しかし,これは$\kappa$-大域被覆性質から直ちに従う:
\begin{lemma}\label{lem:global-cover-gives-code}
 $(W, V)$が$\kappa$-大域被覆性質を満たすなら,任意の$A \subseteq \mu$に対し$A \models T_g$となる$g \in W$が取れる.
\end{lemma}
\begin{proof}
 $V$において$f: [\mathcal{L}]^\kappa \cap W \to \mathcal{L}$を次で定める:
 \[
  f(\Gamma) \defeq \begin{cases}
                    \varphi & (\text{if } \varphi \in \Gamma, A \models \varphi)\\
                    \text{arbitrary} & (\ow).
                   \end{cases}
 \]
 この時,$\mathcal{L}$-論理式を適切に添え字づけておけば,$\kappa$-被覆性質から$g \in W$,$g: [\mathcal{L}]^\kappa \cap W \to [\mathcal{L}]^{<\kappa} \cap W$で$f(\Gamma) \in g(\Gamma)$を満たすものが取れる.
 すると取り方から明らかに$A \models T_g$となるので,補題~\ref{lem:GA-code-suff}より$G_A$は$A$をコードする$(V, \mathbb{P}_g)$-生成集合となる. \qed
\end{proof}

以上から直ちに目標の次の系が得られる:
\begin{corollary}[$\kappa$-大域被覆性に対する$\kappa$-\cc{}強制法の完全性]\label{cor:global-covering-complete}
 $(W, V)$が$\kappa$-大域被覆性質を満たすなら,任意の$x \in V$は$W$のある$\kappa$-\cc{}強制法$\mathbb{P}_x \in W$により付加される.
 特に,$x \in V$は$(W, \mathbb{P}_x)$-生成的となる.
\end{corollary}

\subsection{Bukovsk\'{y}の定理の証明}
これまでの道具立てを使えば,Bukovskyの定理は上で述べた方針をなぞるだけで示せる:
\begin{proof}[Proof of Bukovsk\'{y}'s Theorem]
 $(W, V)$が$\kappa$-大域被覆性質を持つとする.
 記号が重いので,$\mu \defeq (2^{<\kappa})^V$とおく.

 この時,上の系~\ref{cor:global-covering-complete}より$W$における$\kappa$-\cc{}強制法$\mathbb{P}$が存在し,$A \defeq \Pow^V(\mu)$は$(W, \mathbb{P})$-生成的になっている.
 $W[A]$の推移性より$A \subseteq W[A]$だから,結局$\Pow^{W[A]}(\mu) = A$となる.
 また,$(W[G], V)$も$\kappa$-大域被覆性を持つことに注意する.
 実際,$f: \alpha \to \On$を$V$の写像とすれば,$(W, V)$の$\kappa$-大域被覆性より$F \in W$で$F: \alpha \to [\On]^{<\kappa}$かつ$f(\xi) \in F(\xi)$を満たすものが取れ,$F \in W \subseteq W[A]$となる.

 そこで,任意に$x \in V$を任意に取れば,再び系~\ref{cor:global-covering-complete}より$W[A][x]$は$W[A]$の$\kappa$-\cc{}強制拡大になっている.
 しかし,$W[A] \subseteq W[A][x] \subseteq V$より:
 \[
 A = \Pow^{W[A]}(\mu) \subseteq \Pow^{W[A][x]}(\mu)  \subseteq \Pow^V(\mu) = A.
 \]
 よって$\Pow^{W[A]}(\mu) = \Pow^{W[A][x]}(\mu)$を得る.
 すると,系~\ref{cor:2<kappa-cc-tirival}より$W[A] \subseteq W[A][x]$は自明な拡大なので$x \in W[A][x] = W[A]$となる.
 いま$x \in V$は任意だったから$W[A] = V$となる.
 よって$V$は$W$の$\kappa$-\cc{}拡大である. \qed
\end{proof}

\section{$\ZFC$の部分体系への一般化と一意性定理}
これまでの証明を注意深く分析すれば,$\ZFC$よりも弱い体系についてBukovsk\'{y}の定理が成り立つことがわかる.

特に,第一回に定義した以下の体系$\ZFC_\delta^*$に置換公理を加えたモデルについてもBukovsk\'{y}の定理は成り立つ:
\begin{definition}[$\ZFC_\delta^{\leq\kappa}$]
 $\ZFC_\delta^*$の言語は述語記号$\in$に加え定数記号$\delta, \lambda$を持ち,公理は以下で与えられる.
 \begin{enumerate}
  \item $\mathrm{ZC}-\mathrm{Power}$:内包,対,和集合,無限,基礎,整列定理,
  \item 「$\delta$は正則基数」,
  \item ${\leq}\delta$-置換公理:任意の$f: \delta \to V$と集合$A$に対し像$f[A]$が存在.
  \item ${\leq}\kappa$-冪集合公理:任意の集合$A$に対し濃度$\delta$以下の部分集合全体からなる集合$[A]^{\leq\kappa}$が存在.
  \item $\Pow(\power{<\kappa}{2})$が存在する.
  \item 順序数コード公理:任意の集合$A$は順序数$\alpha$とその上の二項関係$E$により$\braket{\trcl(\set{A}), {\in}} \simeq \braket{\alpha, E}$の形でコード出来る.
 \end{enumerate}
 また,$\ZFC^{\leq\kappa} \deffml \quoted{\ZFC^{\leq\kappa}_\kappa + \text{無制限の置換公理}}$,$\ZFC_\delta \deffml \quoted{\ZFC_\delta^{\leq\delta} + \text{冪集合公理}}$と略記する.
\end{definition}

\begin{corollary}
 $W \subseteq V$を共に$\ZFC^{\leq\delta}$の推移的モデルとする.
 この時,$(W, V)$が$\delta$-大域被覆性性質を持つことと$V$が$W$の$\delta$-\cc{}拡大であることは同値.
\end{corollary}
\begin{proof}
 強制法が$\ZFC^{<\kappa}_\delta$の公理を保存する事と,$A = \Pow^V(\power{<\delta}{2})$にMostowski絶対性を適用するのにフルの冪集合は必要ない事などに気を付ければ,これまでの証明が通ることがわかる. \qed
\end{proof}
これから直ちに次が従う.
\begin{lemma}[$\delta$-大域被覆性質を持つモデルの一意性]\label{lem:global-covering-unique}
 $\mu \defeq (2^{<\delta})^+$とおく.
 $W, W' \subseteq V$が十分に大きな$\kappa$について$\ZFC^{\leq\mu}$の推移的モデルで$(W, V)$および$(W', V)$が共に$\delta$-大域被覆性質を持つとする.
 もし$\power{<\mu}{2} \cap W = \power{<\mu}{2} \cap W'$なら$W = W'$となる.
\end{lemma}
\begin{proof}
 $\delta$-大域被覆性質から$V$は$W$および$W'$の$\delta$-\cc{}拡大になっている事がわかる.
 すると補題~\ref{lem:nontriv-2<kappa}の証明から$(W, V)$および$(W', V)$は$\mu$-近似性質を持つ.
 また,明らかに大域被覆性質は上に遺伝するので,これらは$\mu$-大域被覆性質を持ち,$\mu$の正則性より特に$\mu$-被覆性質も持つ.
 また,$\mu$-大域被覆性質より$\mu^{+W} = \mu^{+W'} = \mu^{+V}$も成り立つ.
 したがって前回示した$\ZFC^{\leq\mu}$のモデルの一意性から$W = W'$となる. \qed
\end{proof}

\section*{次回予告}
\href{http://konn-san.com/math/geology-proof-of-ddg.html}{次回}は,いよいよ目標であった$\sDDG$の証明を与えます.

\nocite{Fuchs:2014fj,Usuba:2017fp}
\nocite{Friedman:2016lr}
\printbibliography[title=参考文献]
\end{document}
